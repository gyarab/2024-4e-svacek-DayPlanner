\thispagestyle{empty}
\section*{Úvod}
\hspace{14pt} Plánování času je klíčovým faktorem pro zvýšení produktivity a dosažení osobních cílů. Zaznamenal jsem, že mnoho existujících nástrojů se však zaměřuje buď na jednorázové úkoly, nebo na sledování návyků, přičemž obě oblasti jsou často oddělené. Tato aplikace se snaží spojit tyto dvě disciplíny do jednoho přehledného systému, čímž nabízí ideální orientaci v plánování dne.

Hlavním principem aplikace je blokové plánování, které umožňuje realisticky rozvrhnout čas a předejít přeplněným či neefektivním harmonogramům. Aplikace dělá návyky atraktivní pomocí přehledných statistik a umožňuje uživateli začít s návykem pomalu a postupně v průběhu času se posouvat k tíženému cíli. Tyto důležité principy vycházejí z knihy Atomové návyky od Jamese Cleara.

Aplikace kombinuje časovou osu, která vizualizuje denní plán, notifikační systém pro připomínání úkolů a statistiky ve formě grafů a map, které pomáhají uživateli sledovat jeho pokrok. Díky této integraci nejen podporuje plánování jednorázových akcí, ale zároveň motivuje ke konzistentnímu budování návyků.  

Cílem bylo inovativním přístupem umožnit uživateli nejen plnit všechny úkoly, které má udělat, a nezapomínat na důležité schůzky, ale také mu pomoci s dlouhodobým rozvojem prostřednictvím budování zdravých návyků.