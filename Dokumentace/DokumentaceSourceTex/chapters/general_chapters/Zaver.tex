\thispagestyle{empty}
\section*{Závěr}
\hspace{14pt} Cílem této ročníkové práce bylo vytvořit mobilní aplikaci, která uživatelům umožní efektivně plánovat svůj den pomocí osvědčených metod plánování a propojit opakované úkoly s jednorázovými. Dalším klíčovým bodem bylo zahrnutí funkcionality pro sledování pokroku uživatele v oblasti návyků prostřednictvím statistik. Tento cíl se podařilo úspěšně splnit. Aplikace umožňuje uživatelům plánovat každý den pomocí časové osy, přidávat pokrok k návykům a plnit úkoly, přičemž mohou sledovat statistiky svého pokroku. Dále byla implementována autentizace přes e-mail nebo Google účet pro zajištění bezpečnosti uživatelských dat. Pro zajištění uživatelského komfortu byl přidán tmavý a světlý režim. 

Do budoucna je možné aplikaci vylepšit implementací konceptu "offline-first designu", což znamená, že veškerá data by byla ukládána lokálně i na Cloud, což by umožnilo uživatelům používat aplikaci i offline, a při připojení k internetu by došlo k synchronizaci dat mezi zařízeními. Další možností rozvoje by bylo začlenění komunitních funkcí, například umožnit uživatelům sdílet své úspěchy a návyky s ostatními, což by mohlo podpořit motivaci, například formou budování nových návyků ve skupinách přátel.