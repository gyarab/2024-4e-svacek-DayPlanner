\chapter{Použité technologie}
\hspace{14pt} Při vývoji mobilní aplikace pro Android jsem se rozhodoval mezi programovacími jazyky Java a Kotlin. Oba jsou nativní pro tuto platformu, ale zvolil jsem Javu, protože ji již dobře znám a mám s ní zkušenosti. Uživatelské rozhraní aplikace je napsáno v XML, což je standardní formát pro definování vizuálních prvků v Android aplikacích.  

Jako vývojové prostředí jsem vybral Android Studio, které je speciálně navrženo pro vývoj mobilních aplikací v Javě. Nabízí předpřipravenou strukturu projektu, nástroje pro ladění a integraci s různými službami, což výrazně usnadňuje práci. Součástí tohoto prostředí je také Gradle, moderní systém pro správu sestavení a závislostí. Gradle umožňuje efektivní správu knihoven, optimalizaci sestavování aplikace a zjednodušení celého vývojového procesu. Díky němu lze snadno přidávat externí knihovny, nastavovat různé build konfigurace a využívat pokročilé funkce, jako je cacheování sestavení nebo podpora multimodulové architektury.  

Pro ukládání dat jsem použil dvě odlišné databáze. První je SQLite, což je vestavěná lokální databáze v Androidu. Použil jsem ji především pro ukládání úkolů a uživatelských preferencí, jako je volba tmavého nebo světlého režimu či nastavení notifikací. Dále jsem implementoval Firebase, která se stará o autentizaci uživatelů prostřednictvím e-mailu a Google účtu. Ve Firebase jsou uloženy návyky uživatelů, což umožňuje synchronizaci mezi zařízeními. Tímto způsobem uživatel o svá data nepřijde, ani když si pořídí nový telefon – stačí se pouze přihlásit pod stejným účtem a všechna jeho data se automaticky načtou.
