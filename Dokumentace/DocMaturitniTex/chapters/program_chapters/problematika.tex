\chapter{Řešená problematika}
\hspace{14pt} Tato kapitola pojednává o tom, jaký reálný problém moje aplikace řeší. Jsou zde zmíněny ověřené metody a postupy, ze kterých aplikaci vychází, aby byla pro uživatele co nejvíc nápomocná. Také je potřeba vysvětlit a upřesnit určité termíny, které se v této dokumentaci vyskytují.
\section{Jednorázové akce a návyky}
\hspace{14pt} Každý den – a život obecně – se skládá ze dvou základních typů událostí. Na jedné straně stojí ty, které jsou nám důvěrně známé, stálé a tvoří pevnou součást naší rutiny. Mohou to být každodenní činnosti, jako je cesta do školy či do práce, venčení psa po návratu domů, ranní běh nebo víkendová návštěva sauny. Jsme na ně již zvyklí nebo si na ně začínáme zvykat.

Na druhé straně však vstupují do našeho dne události nové, nečekané a často nepředvídatelné. Ty vybočují z našeho běžného rytmu a vyžadují, abychom na ně reagovali a přizpůsobili se jim. Může jít o neohlášený test z matematiky, návštěvu nového kadeřníka, pracovní schůzi nebo i události, na které se má ročníková práce nevztahuje, jako je například přestěhování se do jiného města či náhlé onemocnění.  
Lze tedy říci, že den se skládá z prvků řádu a chaosu, dvou protikladů symbolizovaných principy Jin a Jang. To, co je známé a stabilní, odpovídá Jangu – představuje pevný řád a strukturu. Naopak vše nové, nečekané a neuchopitelné spadá pod Jin, tedy symbol chaosu. Společně vytvářejí dynamickou rovnováhu, která utváří nejen naše dny, ale i celý život. \textit{„Řád a chaos jsou zobrazeny jako jin a jang ve slavném taoistickém symbolu dvou do sebe zaklesnutých draků. Řád ztělesňuje bílý, maskulinní drak, chaos pak jeho černý, femininní protějšek. Černá tečka v bílém poli a bílá v černém naznačují možnost transformace – zrovna když si říkáme, že svět je stabilní a známý, objeví se něco nového, nějaká nečekaná hrozba. A když se naopak vše zdá ztraceno, může ze zkázy a chaosu povstat nový řád."}\cite{peterson2025} 

\newpage

Co se mé aplikace týče, řád reprezentují návyky a chaos - jednorázové akce (úkoly, testy, schůze, termíny apod.).
\section{Problém současných aplikací}
\hspace{14pt} Dovolím si říct, že nejsem zdaleka jediný, kdo již vyzkoušel nespočet různých aplikací, zabývajících se tématy jako je: plnění úkolů, sledování návyků, připomínky a tak dále. Většinou mi na nich něco chybělo. Vadilo mi, že abych měl přehled o všem, musel jsem mít aplikací několik, a to není praktické. Při navrhování své aplikace jsem vyšel ze zkušeností a snažil se přijít na řešení různých problémů, které jsou zmíněny níže.

Například problém takzvaných to-do listů, ať už papírových nebo digitálních, je ten, že si člověk zpravidla velmi často zapíše na daný den tolik věcí, že je realisticky nikdy nemůže všechny stihnout. Další den udělá to samé a to následně vede k tomu, že se věci stále hromadí, odsouvají a k žádnému pozitivnímu výsledku to nevede. Za prvé člověk nepracuje tak efektivně a za druhé se může mnohdy cítit přehlcený. Přestože za den stihne udělat větší množství úkolů, tak se nemusí dostavit žádný pocit uspokojení, protože v seznamu stále ještě mnoho položek zbývá. Takže místo dobrého pocitu ze sama sebe se dostaví nespokojenost, stres nebo přehlcení.

Také vznikají aplikace, které tento problém řeší tím, že dané akce mají konkrétní datum a časovou délku, což uživateli pomáhá uvědomit si, jak dlouho konkrétní akce trvají, a dokáže si pak den naplánovat realisticky. Tyto aplikace ale pak často neumožňují si dané úkoly odklikávat a fungují spíše jako kalendář. I když uživateli umožňují zakládat si opakované akce, problém je v tom, že přidané položky zase slouží spíše jako přehled toho, co má uživatel před sebou a nemotivuje uživatele k tomu, aby si dané návyky tvořil a zlepšoval se v nich. Také většinou neposkytují žádné statistiky a zpětnou vazbu opakovaných akcí.

Poté je tu celá řada sledovačů návyků. Tyto aplikace se velmi hodí na budování a udržování zdravých a prospěšných návyků, případně zbavování se těch špatných. Většinou ale pak neumožňují si zapisovat akce, které se neopakují a vést si kalendář s důležitými termíny. 
\newpage
Proto jsem se rozhodl vytvořit tuto aplikaci. Netvrdím, že je moje aplikace úplně jedinečná, nicméně poskytuje spojení již zmíněných dvou disciplín do jedné uživatelsky přívětivé aplikace, která adresuje tyto problémy a je navržena tak, že vychází z ověřených metod efektivního plnění úkolů a tvoření návyků.

\section{Přístup mé aplikace k problematice}
\hspace{14pt} V této kapitole je popsáno, jak jsem ke zmíněným problémům přistupoval. Aby bylo zajištěno, že si uživatel nenaplánuje tolik úkolů, že je nebude moci stihnout, nebo si jich naopak naplánuje zbytečně málo, k plánování se v mé aplikaci používá časová osa. Když uživatel přidává novou akci, musí zadat datum, čas začátku úkolu a dobu, jak dlouho se chce danou aktivitou zabývat. Už jen to, kdy si člověk přesně stanoví datum a čas, zvyšuje pravděpodobnost, že člověk úkol opravdu splní. \textit{„Podněty, které mohou spustit návyk, mají širokou škálu forem, ale dvě nejčastější jsou čas a místo. Implementační záměry využívají oba tyto podněty. Obecně řečeno, formát pro vytvoření implementačního záměru je: ``Udělám [TOTO] dne [DATUM] na [MÍSTĚ] v [ČAS].''}\cite{douglas2025}

To, že uživatel zadává dobu trvání, pomáhá zachovat si realistickou představu a zároveň umožňuje využít čas efektivně. Vždy, když uživatel přidá novou akci, všechny existující akce v rámci dne se seřadí. Vizuální výška daných úkolů se odvíjí od toho, jak je úkol časově dlouhý. To znamená, že časově delší akce bude vyšší než časově kratší akce. To je takzvané blokování času. \textit{„Blokování času je technika, kdy přiřazuješ takzvaný časový blog konkrétním úkolům nebo aktivitám. Na rozdíl od to-do listů se při blokování času musíš rozhodnout předem, kolik času nad úkolem strávíš. To ti pomůže lépe se soustředit po dobu práce a zaměřit se jen na daný úkol."}\cite{vos2025}
\newpage 
Uživatel si může jednak přidat jednorázovou akci a jednak návyk, tedy opakovanou akci. U návyku musí zvolit začáteční datum návyku a frekvenci, jak se bude daný návyk opakovat. Úkoly můžeme označit za dokončené a u návyků si můžeme každý den zapisovat pokrok. Kdyby se opakované akce jen plnily nebo neplnily, vznikl by problém, kdyby uživatel například četl deset minut z plánovaných patnácti, návyk by byl nesplněný. V mé aplikaci si ale může zapsat, kolik daného návyku splnil, což má za výsledek to, že není demotivovaný, když místo patnácti minut čte deset. Nebo naopak čte jeden den knihu dvacet minut z cílových patnácti a do aplikace si může zapsat i víc, než je cílová hodnota. To má za účel motivovat uživatele a pomáhá to k vytvoření statistik, které co nejvíce odpovídají realitě.

Uživateli je umožněno zvyšovat si cíl návyku v průběhu času. To znamená, že cíl je například: číst každý den třicet minut. Uživatel ale začne pouze s jednou minutou a postupně si zvyšuje cíl, když už mu současný připadá snazší. Pokud si formujeme nový návyk, doporučuje se používat takzvané pravidlo dvou minut. \textit{„Pravidlo dvou minut říká, že jakýkoli nový návyk by měl být zmenšen na akci, která trvá méně než dvě minuty. Tento přístup činí návyky méně skličujícími a zvyšuje pravděpodobnost, že začnete. To může vypadat následovně: Číst před spaním každou noc se stává Přečíst jednu stránku Dělat třicet minut jógy se stává Vytáhnout podložku na jógu. Tyto malé akce slouží jako "brána k návykům", které vedou k většímu chování, které chcete přijmout. Jakmile začnete, je snazší pokračovat. Klíčem je učinit návyky co nejjednoduššími na začátek, což umožňuje setrvačnosti vás posunout vpřed."}\cite{sobrief_atomic_habits} Uživateli je umožněno začít návyk zlehka a postupně zvyšovat cílovou hodnotu. Dále se také vytváří statistika, díky které uživatel může sledovat, jak se blíží nebo neblíží svému cíli.
